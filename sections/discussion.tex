\section{Discussion}
\label{sec:discussion}

\subsection{Obliviousness and confidentiality}

The BHF was originally motivated by \emph{oblivious} data structures for
encrypted search~\cite{es}.
An oblivious set (or map) is one whose binary representation reveals minimal
information about the underlying data beyond what is accessible through the
defined interface.

The maximum entropy property (\cref{thm:max_entropy}) is central to
obliviousness: the BHF encoding is incompressible (conditioned on its length),
so no side-channel information beyond the encoding length is available to an
adversary.
The encoding length itself reveals only an estimate of the cardinality, and
this leakage can be controlled via the entropy--space tradeoff
(\cref{sec:entropy}).

For Bernoulli maps, the choice of value encoder affects obliviousness.
An optimal (e.g., Huffman) encoder minimizes space but reveals distributional
information about the values.
A uniform-length encoder preserves obliviousness at the cost of higher $\mu$.

\subsection{Comparison with existing data structures}

\paragraph{Bloom filter~\cite{bloom1970,bf_survey}.}
The Bloom filter is a practical implementation of the positive Bernoulli set
($\fnrate = 0$).
It achieves $\log_2(e) / \fprate$ bits per element, which exceeds the
information-theoretic lower bound of $-\log_2 \fprate$ by a factor of
$\log_2 e \approx 1.44$.
The Bloom filter does not support false negatives ($\fnrate > 0$) or maps.

\paragraph{Perfect Hash Filter~\cite{phf}.}
The PHF uses a perfect hash function to achieve $\log_2(e/\fprate)$ bits per
element for positive Bernoulli sets.
This is closer to optimal than the Bloom filter but still exceeds the lower
bound.
The Perfect Map Filter~\cite{pmf} extends this to maps.

\paragraph{Bernoulli Hash Function.}
The BHF achieves the exact lower bound ($-\log_2 \fprate + \mu$ bits per
element) and maximizes entropy.
However, the construction time is exponential in $m$ (expected
$\mathcal{O}(1/p)$ trials), making it a theoretical result that establishes
the achievability of the bound.

\subsection{Practical considerations}

While the BHF construction is exponential-time and thus impractical for large
sets, it serves several purposes:
\begin{enumerate}
    \item It proves that the information-theoretic lower bound is
    \emph{achievable}, not merely a lower bound.
    \item It provides a benchmark against which practical data structures
    (Bloom filters, PHFs) can be compared.
    \item For very small sets ($m \leq 10$), the construction is feasible and
    may be useful in specialized applications such as encrypted search indices.
\end{enumerate}

\subsection{Encrypted search application}

In encrypted search~\cite{es}, a client constructs BHF instances of document
term sets and stores them on an untrusted server.
To search, the client provides the server with a query term and the server
tests membership in each BHF without learning the document contents.
The Bernoulli map variant enables rank-ordered search: each key (term) maps
to a relevance score, and the server can retrieve approximate scores for
query terms.
The maximum entropy and obliviousness properties ensure that the server learns
only what is revealed by the query results.
