\section{Space complexity}
\label{sec:space}

We prove that the BHF achieves the information-theoretic lower bound on space
complexity for both Bernoulli sets and Bernoulli maps, under both predicate
forms.

\subsection{General proof (maps)}

The BHF representation is the tuple $(h_k, b)$ or $(N, t, b)$.
The dominant term is the length of the salt $b$, which the algorithm discovers
by exhaustive search in order of increasing length.

\begin{theorem}[Success probability per trial]
\label{thm:success_prob}
The probability that a candidate salt $b$ yields a valid BHF for a map with
$m$ keys, false positive rate $\fprate$, and mean value encoding length $\mu$
is:
\begin{enumerate}[(i)]
    \item Equality predicate: $p_{\mathrm{eq}} = \fprate^{m-1} / 2^{m\mu}$.
    \item Threshold predicate: $p_{\mathrm{th}} = \fprate^{m} / 2^{m\mu}$.
\end{enumerate}
\end{theorem}
\begin{proof}
See \cref{thm:success_eq,thm:success_thresh}.
\end{proof}

\begin{theorem}[Space complexity of the BHF]
\label{thm:space}
The expected bit length of the BHF asymptotically achieves
\begin{equation}
    -\log_2 \fprate + \mu \;\; \si{bits \per element}
\end{equation}
as $m \to \infty$, for both predicate forms.
\end{theorem}
\begin{proof}
The algorithm searches through candidate salts in order of increasing bit
length.
By \cref{def:mapping}, the \nth candidate maps uniquely to a bit string.
The search terminates at the first success, which follows a geometric
distribution with parameter $p$ (where $p = p_{\mathrm{eq}}$ or
$p = p_{\mathrm{th}}$).

Let $\RV{Q} \sim \geodist(p)$ be the trial at which the first success occurs.
The salt bit length is $\RV{N} \approx \log_2 \RV{Q}$ (a slight overestimate
from dropping the floor).

We approximate $\Expect{\RV{N}}$ via a second-order Taylor expansion of
$\log_2$ around $\Expect{\RV{Q}}$:
\begin{equation}
    \Expect{\RV{N}} \approx \log_2 \Expect{\RV{Q}}
        - \frac{\log_2 e}{\Expect{\RV{Q}}} \Var{\RV{Q}}\,.
\end{equation}
For $\RV{Q} \sim \geodist(p)$, $\Expect{\RV{Q}} = 1/p$ and
$\Var{\RV{Q}} = (1-p)/p^2$.
Substituting:
\begin{equation}
    \Expect{\RV{N}} \approx -\log_2 p + (1 - 1/p) \log_2 e\,.
\end{equation}

\paragraph{Equality predicate.}
With $p = \fprate^{m-1}/2^{m\mu}$:
\begin{align}
    \Expect{\RV{N}}
        &\approx -(m-1)\log_2 \fprate + m\mu
            + (1 - \fprate^{-(m-1)} 2^{m\mu}) \log_2 e\,.
\end{align}
Dividing by $m$ and taking $m \to \infty$:
\begin{equation}
    \frac{\Expect{\RV{N}}}{m} \to -\log_2 \fprate + \mu\,.
\end{equation}

\paragraph{Threshold predicate.}
With $p = \fprate^{m}/2^{m\mu}$:
\begin{align}
    \Expect{\RV{N}}
        &\approx -m\log_2 \fprate + m\mu
            + (1 - \fprate^{-m} 2^{m\mu}) \log_2 e\,.
\end{align}
Dividing by $m$ and taking $m \to \infty$:
\begin{equation}
    \frac{\Expect{\RV{N}}}{m} \to -\log_2 \fprate + \mu\,.
\end{equation}
In both cases, the expected bits per element converges to the information-theoretic lower bound $-\log_2 \fprate + \mu$.
\end{proof}

\subsection{Set as corollary}

\begin{corollary}[Space complexity for sets]
\label{cor:set_space}
For a Bernoulli set ($\mu = 0$), the BHF achieves $-\log_2 \fprate$ bits per
element asymptotically.
\end{corollary}
\begin{proof}
Set $\mu = 0$ in \cref{thm:space}.
\end{proof}

\subsection{Finite-sample correction}

For finite $m$, the exact bits per element are:
\begin{equation}
    \frac{\Expect{\RV{N}}}{m} =
    \begin{cases}
        -\dfrac{m-1}{m}\log_2 \fprate + \mu + \mathcal{O}(1/m)
            & \text{(equality)}\,,\\[8pt]
        -\log_2 \fprate + \mu + \mathcal{O}(1/m)
            & \text{(threshold)}\,.
    \end{cases}
\end{equation}
The equality predicate is slightly more space-efficient for small $m$
(by $\log_2 \fprate / m$ bits per element) due to the adaptive $h_0$.

\subsection{Effect of false negatives}

When $\fnrate > 0$, only $p = (1-\fnrate)m$ elements need to be accepted.
The success probability becomes (for the equality predicate):
\begin{equation}
    p = \frac{\fprate^{p-1}}{2^{p\mu}}\,,
\end{equation}
and the space complexity per element of the \emph{original} set becomes:
\begin{equation}
    \frac{p}{m}\left(-\log_2 \fprate + \mu\right)
    = (1 - \fnrate)\left(-\log_2 \fprate + \mu\right)\,,
\end{equation}
matching the lower bound in \cref{post:map_lb}.
