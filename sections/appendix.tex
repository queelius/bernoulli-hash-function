\appendix
\appendixpage
\addappheadtotoc

\section{Bit length sampler and PMF derivation}
\label{sec:appendix}

\begin{algorithm}[ht]
    \caption{Bit length sampler}
    \label{alg:sampler}
    \SetKwProg{func}{function}{$\colon$}{}
    \KwIn{
        $m$ is the cardinality of the set (or map) to approximate,
        $\fprate$ is the false positive rate,
        $\mu$ is the mean value encoding length.
    }
    \KwOut{
        A random bit length $n$ of an BHF encoding.
    }
    \func{\sampler{$m$, $\fprate$, $\mu$}}{
        Draw a random set $\Set{S}$ of $m$ elements from $\cisb$\;
        \If{$\mu > 0$}{
            Assign random values $v_i$ to each $x_i \in \Set{S}$ with mean
            encoding length $\mu$\;
        }
        $(h_k, b) \gets \MakeBHF(\Set{S}, \fprate, 0)$\;
        \Return $\BL(h_k) + \BL(b) + \mathcal{O}(1)$\;
    }
\end{algorithm}

\begin{theorem}[PMF of the random bit length]
\label{thm:pmf_detailed}
The random bit length $\RV{N}$ of the salt $b$ has the probability mass
function
\begin{equation}
    \PDF{n \Given m, \fprate, \mu}[\RV{N}]
    = q^{2^n - 1}\!\left(1 - q^{2^n}\right)\,,
\end{equation}
where $q = 1 - p$ and $p$ is the per-trial success probability.
\end{theorem}
\begin{proof}
Each candidate salt is an independent Bernoulli trial with success probability
$p$.
The algorithm tests candidates in order of increasing bit length, with $2^n$
candidates of length $n$.

For $\RV{N} = n$ to occur:
\begin{enumerate}
    \item All $2^n - 1$ candidates of length $< n$ must fail.
    Each fails independently with probability $q = 1 - p$, so the joint
    failure probability is $q^{2^n - 1}$.

    \item At least one candidate of length $n$ must succeed.
    All $2^n$ candidates of length $n$ fail with probability $q^{2^n}$, so
    the complementary probability is $1 - q^{2^n}$.
\end{enumerate}
By independence, $\Prob{\RV{N} = n} = q^{2^n - 1}(1 - q^{2^n})$.

\paragraph{Verification that this is a valid PMF.}
The sum telescopes:
\begin{align}
    \sum_{n=0}^{\infty} q^{2^n - 1}(1 - q^{2^n})
    &= \sum_{n=0}^{\infty} \left(q^{2^n - 1} - q^{2^{n+1} - 1}\right) \\
    &= (1 - q) + (q - q^3) + (q^3 - q^7) + (q^7 - q^{15}) + \cdots \\
    &= 1\,.
\end{align}

\paragraph{Alternative derivation.}
Since $\RV{N} = \lfloor \log_2 \RV{Q} \rfloor$ where
$\RV{Q} \sim \geodist(p)$:
\begin{align}
    \Prob{\RV{N} = n}
    &= \Prob{2^n \leq \RV{Q} < 2^{n+1}} \\
    &= \sum_{j=2^n}^{2^{n+1}-1} p(1-p)^{j-1} \\
    &= p \cdot \frac{q^{2^n - 1} - q^{2^{n+1} - 1}}{1 - q}
    = q^{2^n - 1}(1 - q^{2^n})\,.
\end{align}
\end{proof}

\begin{theorem}[Expected encoding size]
\label{thm:expected_size}
The expected salt bit length is
\begin{equation}
    \Expect{\RV{N}} = \sum_{n=1}^{\infty} q^{2^n - 1}
    = q + q^3 + q^7 + q^{15} + \cdots\,,
\end{equation}
where $q = 1 - p$.
\end{theorem}
\begin{proof}
\begin{align}
    \Expect{\RV{N}}
    &= \sum_{n=0}^{\infty} n\, q^{2^n - 1}(1 - q^{2^n}) \\
    &= \sum_{n=0}^{\infty} n\left(q^{2^n - 1} - q^{2^{n+1} - 1}\right)\,.
\end{align}
Expanding the first few terms:
\begin{align}
    \Expect{\RV{N}}
    &= 0 \cdot (1 - q) + 1 \cdot (q - q^3)
        + 2 \cdot (q^3 - q^7) + 3 \cdot (q^7 - q^{15}) + \cdots \\
    &= q + q^3 + q^7 + q^{15} + \cdots
    = \sum_{n=1}^{\infty} q^{2^n - 1}\,.
\end{align}
\end{proof}
