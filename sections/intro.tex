\section{Introduction}
\label{sec:intro}

The Bernoulli set~\cite{bernoulli_sets} and Bernoulli map~\cite{bernoulli_maps}
are abstract data types that model sets and maps with quantifiable random errors.
A Bernoulli set $\ASet{S}$ of a set $\Set{S}$ tests each element independently:
negatives test positive with probability $\fprate$ (false positive rate) and
positives test negative with probability $\fnrate$ (false negative rate), each as
independent Bernoulli trials.
A Bernoulli map $\ASet{M}$ extends this to key-value pairs: the key domain
behaves as a Bernoulli set, and each positive key maps to its associated value.

This paper addresses the \emph{implementation} question: given a finite set or
map, how do we construct a space-optimal Bernoulli set or map with maximum
entropy?
We introduce the \emph{Bernoulli Hash Function} (BHF), a family of hash-based
constructions parameterized by an \emph{acceptance predicate} over hash
outputs.

\subsection*{Contributions}

We make three contributions:

\begin{enumerate}
    \item \textbf{Unified construction.}
    We present a single construction that implements both Bernoulli sets and
    Bernoulli maps.
    The set is the special case where the mean value bit length $\mu = 0$.
    One proof of space-optimality covers both.

    \item \textbf{Generalized acceptance predicate.}
    We generalize the membership predicate from an equality test
    ($h(x \cat b) = h_0$, yielding $\fprate \in \{2^{-k}\}$) to a threshold
    test ($h(x \cat b) \bmod N \leq t$, yielding
    $\fprate \in \{j/N : j = 1,\ldots,N\}$).
    The threshold test provides finer FPR granularity and, crucially,
    eliminates the $\binom{m}{p}$ subset enumeration when $\fnrate > 0$.
    We further introduce an \emph{adaptive} variant that sets $t$ to the
    $p$-th order statistic of the hash residues, eliminating the salt search
    entirely for sets (success probability 1) at the cost of a random FPR.

    \item \textbf{Information-theoretic optimality.}
    We prove that the BHF achieves the information-theoretic lower bound on
    space complexity while maximizing the entropy of its binary representation,
    under both predicate forms and for both sets and maps.
\end{enumerate}

\subsection*{Relation to companion papers}

The algebraic properties of Bernoulli sets (union, intersection, complement,
error propagation) are developed in~\cite{bernoulli_sets}.
The Bernoulli map algebra (composition, domain/codomain error) is
in~\cite{bernoulli_maps}.
The type-theoretic generalization is in~\cite{bernoulli_data_type}.
This paper focuses exclusively on \emph{constructing} optimal implementations
and cites those companions for the abstract theory.

\subsection*{Organization}

\Cref{sec:prelim} establishes notation and recalls the Bernoulli set and map
definitions.
\Cref{sec:shf} introduces the Bernoulli Hash Function family and the
generalized acceptance predicate.
\Cref{sec:construction} gives the construction algorithms.
\Cref{sec:space} proves space optimality.
\Cref{sec:entropy} analyzes the entropy properties.
\Cref{sec:prob_model} summarizes the probabilistic model.
\Cref{sec:operations} describes set operations on BHF instances.
\Cref{sec:discussion} discusses applications, obliviousness, and comparisons
with existing data structures.
