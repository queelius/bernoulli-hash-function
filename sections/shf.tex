\section{The Bernoulli Hash Function family}
\label{sec:shf}

The \emph{Bernoulli Hash Function} (BHF) is a family of constructions that
encode a finite set or map into a compact binary representation using a hash
function and a \emph{salt}---a bit string $b$ discovered by search.
The key idea is an \emph{acceptance predicate} that determines membership.

\subsection{Acceptance predicates}

\begin{definition}[Acceptance predicate]
\label{def:acceptance}
Given a hash function $\hash$, a salt $b \in \cisb$, a modulus $N \in \NatSet$,
and an \emph{acceptance set} $\Set{A} \subseteq \{0, 1, \ldots, N-1\}$, an
element $x$ is \emph{accepted} if and only if
\begin{equation}
    \hash(x \cat b) \bmod N \in \Set{A}\,.
\end{equation}
Under the random oracle assumption (\cref{asm:ro_approx}), the false positive
rate is
\begin{equation}
    \fprate = \frac{\Card{\Set{A}}}{N}\,.
\end{equation}
\end{definition}

Two important special cases arise from the choice of $\Set{A}$:

\paragraph{Equality predicate (classical BHF).}
Set $N = 2^k$ and $\Set{A} = \{h_0\}$ for some $h_0 \in \{0,\ldots,2^k-1\}$
discovered adaptively during construction.
Then $\fprate = 2^{-k}$ and the membership test is
\begin{equation}
    \hash(x \cat b) \bmod 2^k = h_0\,.
\end{equation}
The achievable false positive rates are $\fprate \in \{2^{-k} : k \in \NatSet\}$.

\paragraph{Threshold predicate (generalization).}
Set $\Set{A} = \{0, 1, \ldots, t\}$ for a threshold $t \geq 0$.
Then $\fprate = (t+1)/N$ and the membership test is
\begin{equation}
\label{eq:threshold_test}
    \hash(x \cat b) \bmod N \leq t\,.
\end{equation}
The achievable false positive rates are
$\fprate \in \{j/N : j = 1, \ldots, N\}$, which is a much finer lattice
than the power-of-two constraint of the equality predicate.

\subsection{Success probability per trial}

The probability that a candidate salt $b$ produces a valid BHF depends on the
predicate type.

\begin{theorem}[Success probability---equality predicate]
\label{thm:success_eq}
For the equality predicate with $m$ keys, false positive rate
$\fprate = 2^{-k}$, and mean value encoding length $\mu$ bits, the probability
that a given salt $b$ yields a valid construction is
\begin{equation}
    p_{\mathrm{eq}}(m, \fprate, \mu) = \frac{\fprate^{m-1}}{2^{m\mu}}\,.
\end{equation}
\end{theorem}
\begin{proof}
Under the random oracle assumption, the hash of each key concatenated with
$b$ is independent and uniform.
The first key fixes the target hash $h_0$; each of the remaining $m-1$ keys
must hash to $h_0$, each with probability $\fprate$.
Additionally, for a map, each key's value must match the first $\BL(v_i)$ bits
of a second hash, contributing a factor of $2^{-\BL(v_i)}$ per key.
The product is
\begin{equation}
    \fprate^{m-1} \cdot \prod_{i=1}^{m} 2^{-\BL(v_i)}
    = \frac{\fprate^{m-1}}{2^{m\mu}}\,,
\end{equation}
where $\mu = \frac{1}{m}\sum_{i=1}^{m} \BL(v_i)$ is the mean value bit length.
For sets, $\mu = 0$ and the expression reduces to $\fprate^{m-1}$.
\end{proof}

\begin{theorem}[Success probability---threshold predicate]
\label{thm:success_thresh}
For the threshold predicate with modulus $N$, threshold $t$, $m$ keys, and mean
value encoding length $\mu$ bits, the probability that a given salt $b$ yields
a valid construction is
\begin{equation}
    p_{\mathrm{th}}(m, \fprate, \mu) = \frac{\fprate^{m}}{2^{m\mu}}\,.
\end{equation}
\end{theorem}
\begin{proof}
With the threshold predicate, the acceptance set $\Set{A} = \{0,\ldots,t\}$ is
fixed \emph{before} examining the salt.
Each of the $m$ keys must independently hash into $\Set{A}$, each with
probability $\fprate = (t+1)/N$.
This gives $\fprate^m$ (not $\fprate^{m-1}$, since no key is ``free'' to fix
the target).
The value-matching factor contributes $2^{-m\mu}$ as before.
\end{proof}

\begin{remark}
The equality predicate has a factor-of-$\fprate^{-1}$ advantage in success
probability because one key adaptively determines $h_0$.
In the space complexity (\cref{sec:space}), this manifests as an additive
$\log_2 \fprate / m$ bits per element, which vanishes as $m \to \infty$.
Both predicates achieve the same asymptotic optimum.
\end{remark}

\subsection{FPR granularity}

\begin{theorem}[FPR granularity]
\label{thm:fpr_granularity}
With modulus $N$, the threshold predicate achieves any false positive rate in
the lattice
\begin{equation}
    \fprate \in \left\{\frac{j}{N} : j = 1, 2, \ldots, N\right\}\,,
\end{equation}
whereas the equality predicate achieves only
$\fprate \in \{2^{-k} : k \in \NatSet\}$.
\end{theorem}

This is immediate from the definitions.
For example, with $N = 100$, one can achieve $\fprate = 0.01, 0.02, \ldots, 1.00$, whereas the equality predicate can only achieve $\fprate = 0.5, 0.25, 0.125, \ldots$

\subsection{Simplified search for $\fnrate > 0$}

When the false negative rate $\fnrate > 0$, only $p = (1-\fnrate)m$ of the
$m$ elements need to be accepted.
Under the equality predicate, the algorithm must enumerate all $\binom{m}{p}$
subsets of size $p$ and check whether each subset collides.
Under the threshold predicate, the search is dramatically simpler.

\begin{theorem}[Search complexity for $\fnrate > 0$]
\label{thm:fnr_search}
Under the threshold predicate, testing whether a candidate salt $b$ admits at
least $p$ of $m$ elements requires $\mathcal{O}(m)$ hash evaluations: simply
count how many elements satisfy $\hash(x \cat b) \bmod N \leq t$.

Under the equality predicate, the same test requires
$\mathcal{O}\!\left(\binom{m}{p} \cdot m\right)$ hash evaluations in the worst
case, since each candidate $h_0$ (determined by each $p$-subset) must be
checked against all elements.
\end{theorem}
\begin{proof}
For the threshold predicate, the acceptance set $\Set{A}$ is fixed.
For each candidate salt $b$, we compute $\hash(x_i \cat b) \bmod N$ for
$i = 1,\ldots,m$ and count the number that fall in $\Set{A}$.
If at least $p$ do, the salt succeeds.
This is $\mathcal{O}(m)$ per candidate salt.

For the equality predicate, the target $h_0$ is not known a priori.
For each $p$-element subset, the first element determines $h_0$, and the
remaining $p-1$ elements must match.
Since there are $\binom{m}{p}$ such subsets, the worst-case cost per candidate
salt is $\mathcal{O}\!\left(\binom{m}{p} \cdot m\right)$.
\end{proof}

\subsection{Adaptive threshold}

An intermediate strategy is the \emph{adaptive threshold}: given a salt $b$,
compute all $m$ hash values $\hash(x_i \cat b) \bmod N$, sort them, and set
$t$ to be the $p$-th smallest value (the $p$-th order statistic).
Then exactly $p$ elements are accepted.
The resulting FPR is the random variable
\begin{equation}
    \fprate = \frac{t + 1}{N}\,,
\end{equation}
which is concentrated around the target FPR as $N$ grows.
This approach combines the $\mathcal{O}(m \log m)$ simplicity of the threshold
test with the adaptive efficiency of the equality test, at the cost of storing
the threshold $t$ as part of the representation.
