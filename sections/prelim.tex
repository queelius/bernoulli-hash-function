\section{Preliminaries}
\label{sec:prelim}

\subsection{Bit strings and encoding}

The binary set is $\BitSet = \{0,1\}$.
The set of all bit strings of length $n$ is $\cisbn{n}$, with
$\Card{\cisbn{n}} = 2^n$.
The countably infinite set of all finite bit strings is
$\cisb = \bigcup_{n=0}^{\infty} \cisbn{n}$.
The bit length of an object $x$ is denoted $\BL(x)$.

\begin{definition}[Bit string--natural number bijection]
\label{def:mapping}
Let $\cisb$ and $\NatSet$ have the bijection
\begin{equation}
    (b_1 \, b_2 \, \cdots \, b_m) \longleftrightarrow 2^m + \sum_{j=1}^{m} 2^{m-j} b_j\,.
\end{equation}
We denote the image of a bit string (or natural number) $x$ under this
bijection by $x'$.
A natural number $n$ maps to a bit string of length $\lfloor \log_2 n \rfloor$.
\end{definition}

\subsection{Hash functions and the random oracle}

\begin{definition}[Hash function]
A \emph{hash function} $\hash \colon \cisb \to \cisbn{n}$ maps arbitrary-length
bit strings to fixed-length bit strings of length $n$.
\end{definition}

\begin{definition}[Random oracle]
\label{def:randomoracle}
A \emph{random oracle} $\ro \colon \cisb \to \cisbn{\infty}$ is a theoretical
hash function whose output is uniformly distributed over its range for every
unique input.
For any prefix length $k$, the first $k$ bits of $\ro(x)$ are uniformly
distributed over $\cisbn{k}$.
\end{definition}

\begin{assumption}[Random oracle approximation]
\label{asm:ro_approx}
The hash function $\hash$ approximates a random oracle $\ro$ and has a space
complexity of $\mathcal{O}(1)$.
\end{assumption}

\subsection{Bernoulli sets}
\label{sec:prelim_bset}

We recall the essential definitions from~\cite{bernoulli_sets}.

\begin{definition}[Bernoulli set]
\label{def:bernoulli_set}
Let $\Set{S} \subseteq \Set{U}$ be a finite set drawn from a universe
$\Set{U}$.
A \emph{Bernoulli set} $\ASet{S}$ of $\Set{S}$ with false positive rate
$\fprate$ and false negative rate $\fnrate$ satisfies:
\begin{enumerate}[(i)]
    \item For each $x \in \Set{S}$, independently,
    $\Prob{x \notin \ASet{S}} = \fnrate$.
    \item For each $x \in \Set{U} \setminus \Set{S}$, independently,
    $\Prob{x \in \ASet{S}} = \fprate$.
\end{enumerate}
A Bernoulli set with $\fnrate = 0$ is a \emph{positive} Bernoulli set,
denoted $\PASet{S}$.
A Bernoulli set with $\fprate = 0$ is a \emph{negative} Bernoulli set,
denoted $\NASet{S}$.
\end{definition}

\begin{postulate}[Space lower bound for sets]
\label{post:set_lb}
The optimal space complexity for a Bernoulli set over a countably infinite
universe is
\begin{equation}
    -(1 - \fnrate) \log_2 \fprate \;\; \si{bits \per element}\,.
\end{equation}
\end{postulate}

\subsection{Bernoulli maps}
\label{sec:prelim_bmap}

We recall the essential definitions from~\cite{bernoulli_maps}.

\begin{definition}[Bernoulli map]
\label{def:bernoulli_map}
Let $\Set{M} \subseteq \Set{X} \times \Set{Y}$ be a finite map (set of
key-value pairs where each key appears at most once).
A \emph{Bernoulli map} $\ASet{M}$ of $\Set{M}$ with false positive rate
$\fprate$ and false negative rate $\fnrate$ satisfies:
\begin{enumerate}[(i)]
    \item The key domain of $\ASet{M}$ is a Bernoulli set of the key domain
    of $\Set{M}$ with rates $\fprate$ and $\fnrate$.
    \item For each positive key $x$, $\Find(\ASet{M}, x)$ returns the correct
    value $\Set{M}[x]$.
    \item For each false-positive key $x$, $\Find(\ASet{M}, x)$ returns a
    value that is effectively random.
\end{enumerate}
\end{definition}

\begin{postulate}[Space lower bound for maps]
\label{post:map_lb}
The optimal space complexity for a Bernoulli map over a countably infinite key
universe, with mean value encoding length $\mu$ bits, is
\begin{equation}
    -(1 - \fnrate) \log_2 \fprate + (1-\fnrate)\mu \;\; \si{bits \per element}\,.
\end{equation}
\end{postulate}

\begin{remark}
Setting $\mu = 0$ in \cref{post:map_lb} recovers the set lower bound
(\cref{post:set_lb}).
This duality---the set as a degenerate map---is the organizing principle of
this paper.
\end{remark}
